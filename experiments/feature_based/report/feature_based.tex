\documentclass[10pt]{report}

\usepackage[margin=2.5cm]{geometry}
\usepackage{tikz}
\usetikzlibrary{shapes,arrows}
\usepackage{amsmath}
\usepackage{array}
\usepackage{subfigure}
\usepackage[colorlinks=true,linkcolor=blue]{hyperref}
\usepackage[all]{hypcap}
\usepackage{lscape}
\usepackage{multirow}
\usepackage[utf8]{inputenc}
\usepackage[T1]{fontenc}
\usepackage{algorithmic}
\usepackage{algorithm}
\usepackage{rotating}
\setlength{\parindent}{0in}

\title{Feature based classifier for Astronomical time series}

\date{}

\begin{document}
	\maketitle
	
	\section{Introduction}
	A variety of features from both the time and frequency domain will be compared in their effectiveness for the automatic classification of simplified Astronomical time series. Classification was carried out with the extracted features on the Weka software suite on a variety of generic supervised learning algorithms.
	
	\section{Features}
	\begin{itemize}
		\item Wavelets
		\item LS features
		\item Amplitude percentiles
		\item Skew
		\item 
	\end{itemize}	
	
			
\end{document}
