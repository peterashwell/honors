\documentclass[12pt]{beamer}

\usepackage{subfigure}
\usepackage{caption}
\usepackage{graphicx}
%\usepackage{natbib}
%\usepackage{bibentry}
\usepackage{biblatex}
\usepackage{amsmath}
\usepackage{multirow}

\bibliography{refs}

\mode<presentation>
\subtitle{Emden R. Gansner, Yehuda Koren \\ \emph{AT\&T Labs - Research}}
\title{Improved Circular Layouts}
\date{}
\author{Peter Ashwell}

%\newcommand{\newblock}{}

\begin{document}
\maketitle

\begin{frame}
	\frametitle{Overview and Circular Layouts}
	\begin{figure}
	  \centering
	  \subfigure {	
	  	\includegraphics[width=120px]{poorlayout.png}
	  }
	  \subfigure {
	  	\includegraphics[width=120px]{three_improvements.png}
	  }
  	%\caption{Wall painting in Chaco Canyon, NM USA}
	\end{figure}
	Proposes algorithms for 3 improvements to circular layouts
	\begin{enumerate}
		\item Edge crossing reduction by edge length minimisation
		\item External edge routing
		\item Edge bundling
	\end{enumerate}

\end{frame}

\begin{frame}
	\frametitle{Edge length minimisation}
	This paper aims to improve node placement by minimising the sum of the edge lengths between the nodes:
	\begin{center}
		Smaller edge length sum $\rightarrow$ Less clutter \\
	\end{center}
	Utilising force directed layouts a la Tutte and Hall, and minimising a cost function:
	\begin{align*}
		\min{x,y}\sum\limits{\langle i,j \rangle \in E}(x_i - x_j)^2 + (y_i - y_j)^2 
	\end{align*}
	The sum of all edge lengths. This is done using a process called \emph{mean iteration}.
\end{frame}

\begin{frame}
	\frametitle{Mean iteration}
	Fix all node positions except 1.
	Place the node at the mean position (barycenter) of nodes it is joined to (minimises edge length):
	\begin{equation}
		(x_i, y_i) \leftarrow \left. \frac{\sum_{j \in N(i)}(x_j,y_j)}{||N(i)||} \right.
	\end{equation}
	Project it back out onto the unit circle
	\begin{equation}
		(x_i, y_i) \leftarrow \left. \frac{(x_i,y_i)}{||(x_i, y_i||} \right.
	\end{equation}
	The nodes are redistributed evenly (preserving angular ordering) at regular iterations to prevent the algorithm from placing them all in the same location (the global minimum).
	
\end{frame}

\begin{frame}
	\frametitle{Mean iteration}
	The second and second graphs represent equations 1 and 2 respectively. \\
	\begin{figure}
		\includegraphics[width=300px]{meaniteration.eps}
	\end{figure}
	In experiments the authors use the \textbf{median} instead of the mean in equation 1, a change they claim gives better results. Algorithm complexity with either is $O(n + |E|)$ ($n$, $|E|$ number of vertices, edges)
	
\end{frame}

\begin{frame}
	\frametitle{Exterior routing}
	External routing can improve readability by using the external face of the circle (see right graph)
	\begin{figure}
		\includegraphics[width=220px]{externalrouting.png}
	\end{figure}

	The authors propose a dynamic programming algorithm to choose which edges to place outside the circle.
\end{frame}

\begin{frame}
	\frametitle{Dynamic programming for optimal exterior routing}
	\footnotesize {
	\begin{align*}
		p_{i,i+1} &= w_{i,i+1} \quad i = 0, \ldots, n - 2 \\
		p_{i,j} &= w_{i,j} + max_{i<k<j}\{p_{ik} + p_{kj}\} \quad i = 0,\ldots,n-3, i + 1 < j < n
	\end{align*}
	}
	\begin{figure}
		\includegraphics[width=200px]{circlesplit.eps}
	\end{figure}
	$p_{i,j}$ is a recursive expression for the maximum number of non-crossing edges between nodes $i$ and $j$. \\
	$w_{i,j}$ are weights assigned to the edges to get a better layout (e.g. number of crossings with other edges).
\end{frame}

\begin{frame}
	\frametitle{Edge bundling}
	\begin{figure}
		\includegraphics[width=300px]{edgeclustering.png}
	\end{figure}
	\begin{enumerate}
		\item Identify collections of edges whose bundling will save ink (through clustering)
		\item Compute the points $M_{1}$ and $M_{2}$ whose positions reduce the length of the edge bundles
		\item Join the edges along $M_{1}$,$M_{2}$
		\item Use a bezier curve representation of the short ends to make the result more readable
	\end{enumerate}
\end{frame}

\begin{frame}
	\frametitle{Definition of ink saved through merging edges}
	
	\begin{figure}
		\includegraphics[width=200px]{step23bundling.png}
	\end{figure}
	
	\footnotesize {
	Optimal positions of $M_1$ and $M_2$ (found using numerical methods):
	\begin{equation}
		(M_1, M_2) = \underset{M_{1},M_{2}}{\operatorname{argmax}}\sum\limits_{p \in S}||M_1 - p|| + ||M_1 - M_2|| + \sum\limits_{p \in T} ||M_2 - p||
	\end{equation}
	}
	\footnotesize {
	Ink saved: sum of old edge lengths - sum of new edge lengths passing through optimal $M_1$ and $M_2$
	\begin{equation}
		\sum\limits_{v_j,u_j \in Q}||v_j -u_j|| - (\sum\limits_{p \in S}||M_1 - p|| + ||M_1 - M_2|| + \sum\limits_{p \i T}||M_2 - p||
	\end{equation}
	}
\end{frame}

\begin{frame}
	\frametitle{Clustering of edges through ink saving measure}
	Using this definition of ink saving for an edge cluster, the authors propose a hierarchical clustering algorithm:
	\begin{enumerate}
		\item Place each edge in its own cluster
		\item Merge clusters that will give the most ink saving and that \textbf{do not have crossing edges}
		\item Repeat at step 1
	\end{enumerate}
	\begin{figure}
		\includegraphics[width=250px]{clustering.eps}
	\end{figure}
	The complexity of this clustering algorithm is $O(|E|^2)$ where $|E|$ is the number of edges.
\end{frame}

\begin{frame}
	\frametitle{Evaluation}
		
	\scriptsize {
		\begin{table}
		\centering
		\begin{tabular}{|c|lll|lll|} \hline
		\multirow{2}{*}{Graph set} & \multicolumn{3}{c|}{No exterior routing} & \multicolumn{3}{c|}{Exterior routing} \\ \cline{2-7}
					& C & M & MC & C & M & MC \\ \hline
		Rome 20 - 29 nodes & 7.18 & 8.01 & 5.51 & 0.83 & 0.68 & 0.46 \\
		Rome 80 - 89 nodes & 167.29 & 161.84 & 130.41 & 69.84 & 58.53 & 46.73 \\
		Random avg deg 4 & 1337.68 & 1186.50 & 1048.19 & 838.42 & 714.08 & 627.06 \\
		Random avg deg 7 & 6979.42 & 6843.71 & 6210.86 & 5216.34 & 5126.31 & 4594.56 \\ \hline
		\end{tabular}
		\caption{Average number of edge crossings}
	\end{table}
	}	
	
	\scriptsize {
	\begin{table}
	\centering
	\begin{tabular}{|c|lll|lll|} \hline
		\multirow{2}{*}{Graph set} & \multicolumn{3}{c|}{No exterior routing} & \multicolumn{3}{c|}{Edge bundling} \\ \cline{2-7}
					& C & M & MC & C & M & MC \\ \hline
		Rome 20 - 29 nodes & 17.16 & 15.49 & 15.50 & 13.23 & 12.52 & 12.54 \\
		Rome 80 - 89 nodes & 53.99 & 46.21 & 45.14 & 36.31 & 33.63 & 33.03 \\
		Random avg deg 4 & 124.08 & 109.29 & 107.96 & 72.04 & 68.20 & 66.77 \\
		Random avg deg 7 & 273.35 & 260.37 & 254.79 & 207.10 & 198.32 & 194.59 \\ \hline
	\end{tabular}
	\caption{Average sum of edge lengths in graph (ink used)}
	\end{table}
	}
	\textbf{M} is the \emph{median} iteration algorithm. \textbf{C} is the \emph{circo} layout algoritm used in graphviz. \textbf{MC} is mean iteration combined a post-processing step taken from \emph{circo}
\end{frame}	

\begin{frame}
	\frametitle{Conclusion}
	\begin{enumerate}
		\item Edge routing is very effective for reducing edge crossings, doing so from between 90-50\% (less for larger graphs)
		\item Edge bundling is quite good at reducing ink used, from between 30-40\%. Consistent per graph size.
		\item Median iteration is slightly better at reducing crossings, and is also better for ink used, even without bundling.
	\end{enumerate}
\end{frame}

\end{document}
