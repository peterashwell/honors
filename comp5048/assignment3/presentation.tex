\documentclass{beamer}
\usepackage{amsmath}
\usepackage{amsfonts}
\usepackage{amssymb}
\usepackage{graphicx}

%Darmstadt
\usepackage[utf8x]{inputenc}
\usetheme{Frankfurt}
\setbeamertemplate{navigation symbols}{}

\title{Data Clustering: A Review\\A.K Jain, M.N. Murty and P.J. Flynn}
\author{Peter Ashwell, Alex Batten, Adam Petrovic, David Rizzuto}
\date{\today}

\begin{document}

\frame{
\titlepage
}

\section{Introduction}
    %You can put a title in the subsection argument if you want a sub title.
    \subsection{}
        \frame
        {
            \frametitle{Aim of the Paper}
            This paper aims to survey and review known and emerging clustering methods (as of 2000), and compare them to show the strengths and weaknesses of each method.

            The authors also attempt to demonstrate some applications for clustering, and how to deal with common issues faced when clustering.
        }

\section{Structure}
    \subsection{}
        \frame
        {
            \frametitle{Structure of the Paper}
                \begin{enumerate}
                    \item Introduction
                    \item Background, terms and definitions
                    \item Pattern representation, feature extraction/selection
                    \item Similarity measures
                    \item Clustering techniques, both common and emerging methods
                    \item Applications
                \end{enumerate}
        }

\section{Main Ideas}
    \subsection{}
        \frame
        {
			\frametitle{Hierarchical Clustering}
			\begin{block}{Agglomerative Clustering}
				Start with every data point in its own cluster and merge until a threshold is met
			\end{block}
			
			\begin{figure}
				\centering
				\includegraphics[width=60mm]{hierarchical.png}
				\caption{Measures used in merging clusters}
			\end{figure}
		}

        \frame
        {
            \frametitle{Partitional Clustering 1: Discrete Methods}
            \begin{itemize}
                \item Square-Error Algorithms
                    \begin{itemize}
                        \item k-means/fuzzy k-means
                        \item ISODATA
                    \end{itemize}
                \item Graph-Theoretic
                    \begin{itemize}
                        \item MST/Delauney-Triangulation based
                        \item Nearest Neighbour
                    \end{itemize}
            \end{itemize}

            \includegraphics[width=0.5\textwidth]{partition_2.png}
            \includegraphics[width=0.5\textwidth]{partition_1.png}
            \begin{center}
                {\sc Left:} Fuzzy Clustering, {\sc Right:} MST-based clustering
            \end{center}
        }

        \frame
        {
            \frametitle{Partitional Clustering 2: Optimization/Search Methods}
                \begin{itemize}
                    \item Heuristic Methods
                        \begin{itemize}
                            \item simulated annealing/gradient descent
                        \end{itemize}
                    \item Genetic Algorithms
                    \item Neural-Networks
                    \item Mixture Resolving
                        \begin{itemize}
                            \item Expectation Maximization
                            \item Nonparametric Methods
                        \end{itemize}
                \end{itemize}
                \begin{center}
                    \includegraphics[width=0.5\textwidth]{partition_3.png}\\
                    {\sc Above:} Clustering as a global optimization problem
                \end{center}
        }

        \frame
        {
            \frametitle{Applications 1}
			\begin{itemize}
				\item Image segmentation and processing
				\item Object and character recognition
			\end{itemize}

			\begin{figure}
				\includegraphics[width=100mm]{landsat.png}
				\caption{Segmenting of satellite imagery reveals water bodies}
			\end{figure}
		}

		\frame
		{
			\frametitle{Applications 2}
			\begin{itemize}
				\item Information Retrieval
				\item Data mining
			\end{itemize}
			\begin{figure}
				\includegraphics[width=90mm]{unsupervised_document.png}
				\caption{Clusters of words that are similar in terms of their context of use}
			\end{figure}
		}

\section{Highlights}
    \subsection{}
        \frame
        {
			\frametitle{General Highlights}
			\begin{itemize}
				\item Introductory language, useful for a general scientific audience
				\item Diverse coverage of practical applications
				\item Wholistic approach to clustering, from domain knowledge, feature selection through to algorithms and applications
			\end{itemize}
        }   

        \frame
        {
	        \frametitle{Specific Highlights}
			\begin{itemize}
				\item Good section on the users dilemma: questions to ask in trying to apply clustering
				\item Comparison of clustering algorithms for large data sets
				\item Texture segmentation using Gabor filter
			\end{itemize}
			\begin{figure}
				\includegraphics[width=80mm]{texture.png}
				\caption{Textures automatically segmented using k-means with Gabor filter features}
			\end{figure}
		}

\section{Drawbacks}
    \subsection{}
        \frame
        {
            \frametitle{Main Drawbacks of the Paper}
            \begin{block}{Objective Functions}
                States that many clustering algorithms optimize the ``squared error'' function, which leads to globular clusters. 
                
                However other objective functions are not mentioned or explored.
            \end{block}
            \pause
            \begin{alertblock}{Comparison of Techniques}
                Some non-mainstream techniques are simply described, and not compared against other methods.
                
                For example: NN-clustering and Nonparametric methods for density-based clustering
            \end{alertblock}
        }
\section{Conclusion}
    \subsection{}
        \frame
        {
            \frametitle{Conclusion}
            In conclusion this paper this paper is a good high level introduction to clustering circa 2000. It covers all known techniques, and demonstrates their possible uses.

            However the comparisons are at times incomplete, and the level of detail is not enough for anyone with serious interest in the field to be able to choose a clustering method for their application.
        }
\end{document}
